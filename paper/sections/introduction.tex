%!TEX root = ../ycac2vec.tex

\subsection{Word embeddings}
% lit review on word embeddings
Probabilistic models such as Latent Dirichlet Allocation \cite{Blei2003} are standard tools for analyzing text data. However, such models use bag-of-words representations. Therefore they extract meaning from word co-occurrance counts on the document level, and ignore the sequential nature of language. Syntax, punctuation, and grammar are intrinsically sequential, and a good model of natural language or tree structures should be able to capture both information from co-occurrance counts and information across time. While Latent Dirichlet Allocation has been used to model music (\textbf{Q:citation?}), music is inherently sequential. For in-depth analysis of style in music, we need embedding models.

Word embeddings, or real-valued vectors representing words in a vocabulary, were first introduced by \cite{Bengio2003} but popularized by \cite{Mikolov2013a}. Such models typically have a log-bilinear form \cite{Mnih2007}, and are trained using negative sampling with a fixed size context window \cite{Mikolov2013a}. This is equivalent to matrix factorization of a shifted pointwise mutual information word-context matrix \cite{Levy}. To see this, construct a co-occurrance matrix of words, where the rows and columns are the words in the vocabulary. Each entry $i, j$ in the matrix is the count of how many times word $i$ (e.g. `dog') ocurred in the context of word $j$ (e.g. `the'). Word embedding models such as the skip-gram model (word2vec) can be viewed as performing singular value decomposition on a transformed version of this matrix.  Compositional word embeddings for learning paragraph or document embeddings have also been proposed \cite{Le2014,Dai2015}. However, \cite{Levy2015a} suggests that much of the success of these types of distributed representations of words is due to the tricks needed to train such models such as noise contrastive estimation. \cite{Arora2015} and (todo: cite arora 2016 polysemy, http://arxiv.org/pdf/1601.03764.pdf) show that word embedding models can also be thought of as random walks in the embedding space.

% lit review on word embeddings applied to things other than natural language
As useful models of discrete data, word embedding models are starting to be used in domains outside of natural language. For example, \cite{Asgari2015} embed protein sequences for classification, and \cite{Guardia-Sebaoun2015} develop an embedding model to build a recommendation system (for example, for recommending movies to users).
There exists some prior work on applying word embedding models to music. \cite{Name2015} trained an embedding model on a corpus of 200 rock songs for the task of recommendating chords to composers.

\subsection{Quantitative stylistic analysis of music}