%!TEX root = ../ycac2vec.tex

As shown in Figure~\ref{fig:whole_corpus_circle_fifths}, we can see that the circle of fifths emerges from the structure of the learned vector space of chords in classical music.
This is not an intuitive result at first.
However, to see that this is a reasonable result of applying the skip-gram word2vec model (\textbf{or CBOW, whichever you used in gensim}), consider the log-likelihood of the model.
Gradients of the log-likelihood with respect to the embeddings are used to train the model.
The log-likelihood means the model will maximize the probability of correctly classifying the context given the training example.
If the model assigns too high a probability to correct contexts, it will be overconfident on other (incorrect) contexts, and the derivative of the log-likelihood will push the embeddings further apart.
But if the model assigns too low a probability to correct contexts, the gradient of the log-likelihood will flip signs and pull the embeddings closer together.
The minimal geometric structure that minimizes these constraints is a circle.
We thus expect a circle from the principal components of highly stable embeddings (such as the embeddings of chords in the circle of fifths).
To see why the circle of fifths respects the ordering, we consider the context window (\textbf{(size 5? in our case)}).
Chords in the circle of fifths occur in each others' contexts, but usually only nearest neighbors (e.g. it is rare to see C major followed by B major).
Therefore C major and G major occur in each others' contexts and will be pushed closer together during training.
But they will be pushed apart from their non-nearest-neighbors (such as B major).
This shows they will respect the ordering apparent in classical music where common practices such as counterpoint result in transitions prevalent on the circle of fifths.